% Sebastian Paaske Tørholm's Resume
% Created: 7 Nov 2009
% Last Modified: 05 Dec 2011

\documentclass[11pt,oneside,a4paper]{article}
\usepackage[utf8]{inputenc}
\usepackage{geometry}
\usepackage[T1]{fontenc}

\pagestyle{empty}
\geometry{letterpaper,tmargin=1in,bmargin=1in,lmargin=1in,rmargin=1in,headheight=0in,headsep=0in,footskip=.3in}

\setlength{\parindent}{0in}
\setlength{\parskip}{0in}
\setlength{\itemsep}{0in}
\setlength{\topsep}{0in}
\setlength{\tabcolsep}{0in}

% Name and contact information
\newcommand{\name}{Sebastian Paaske Tørholm}
\newcommand{\addr}{Vinhaven 51, 1.mf, 2500 Valby}
\newcommand{\phone}{+45\,22\,85\,56\,23}
\newcommand{\email}{sebbe@diku.dk}


%%%%%%%%%%%%%%%%%%%%%%%%%%%%%%%%%%%%%%%%%%%%%%%%%%%%%%%%%
% New commands and environments

% This defines how the name looks
\newcommand{\bigname}[1]{
	\begin{center}\fontfamily{phv}\selectfont\Huge\scshape#1\end{center}
}

% A ressection is a main section (<H1>Section</H1>)
\newenvironment{ressection}[1]{
	\vspace{4pt}
	{\fontfamily{phv}\selectfont\Large#1}
	\begin{itemize}
	\vspace{3pt}
}{
	\end{itemize}
}

% A resitem is a simple list element in a ressection (first level)
\newcommand{\resitem}[1]{
	\vspace{-4pt}
	\item \begin{flushleft} #1 \end{flushleft}
}

% A ressubitem is a simple list element in anything but a ressection (second level)
\newcommand{\ressubitem}[1]{
	\vspace{-1pt}
	\item \begin{flushleft} #1 \end{flushleft}
}

% A resbigitem is a complex list element for stuff like jobs and education:
%  Arg 1: Name of company or university
%  Arg 2: Location
%  Arg 3: Title and/or date range
\newcommand{\resbigitem}[3]{
	\vspace{-5pt}
	\item
	\textbf{#1}---#2 \\
	\textit{#3}
}

% This is a list that comes with a resbigitem
\newenvironment{ressubsec}[3]{
	\resbigitem{#1}{#2}{#3}
	\vspace{-2pt}
	\begin{itemize}
}{
	\end{itemize}
}

% This is a simple sublist
\newenvironment{reslist}[1]{
	\resitem{\textbf{#1}}
	\vspace{-2pt}
	\begin{itemize}
}{
	\end{itemize}
}



%%%%%%%%%%%%%%%%%%%%%%%%%%%%%%%%%%%%%%%%%%%%%%%%%%%%%%%%%
% Now for the actual document:

\begin{document}

\fontfamily{ppl} \selectfont

% Name with horizontal rule
\bigname{\name}

\vspace{-8pt} \rule{\textwidth}{1pt}

\vspace{-1pt} {\small\itshape \addr \hfill \phone; \email}

\vspace{8 pt}




%%%%%%%%%%%%%%%%%%%%%%%%
\begin{ressection}{Uddannelse}

	\begin{ressubsec}{Københavns Universitet}{Bachelor}{Datalogi med matematik som bifag}
		\ressubitem{Igangværende, startet 2006.}
		\ressubitem{Gennemsnit indtil videre: 10,11.} % XXX: Outdated, probably.
	\end{ressubsec}

\end{ressection}


%%%%%%%%%%%%%%%%%%%%%%%%
\begin{ressection}{Arbejde}
	\begin{ressubsec}{Studentermedhjælper}{Jobindex}{Februar 2012 -- }
		\ressubitem{Udvikling på siden såvel som interne systemer, hovedsageligt i Perl.}
	\end{ressubsec}

    \begin{ressubsec}{Instruktor}{Københavns Universitet}{November 2009 -- Juni 2012}
        \begin{reslist}{``Algoritmer og Datastrukturer''}
            \ressubitem{2 ansættelser: April 2012 -- Juni 2012, April 2011 -- Juni 2011}
            \ressubitem{Kursus i essentielle algoritmer og datastrukturer.}
        \end{reslist}
        \begin{reslist}{``Introduktion til Programmering''}
            \ressubitem{2 ansættelser: September 2011 -- November 2011, September 2009 -- November 2009}
            \ressubitem{Introduktionskursus i funktionel programmering i Standard ML.}
        \end{reslist}
        \begin{reslist}{``Objektorienteret Programmering og Design''}
            \ressubitem{November 2009 -- Januar 2010}
            \ressubitem{Introduktionskursus i grundlæggende objektorienteret programmering og design med udgangspunkt i Java.}
        \end{reslist}
    \end{ressubsec}

%	\begin{ressubsec}{Instruktor i kurset ``Introduktion til Programmering''}{Københavns Universitet}{September 2011 -- November 2011}
%		\ressubitem{Undervisning i funktionel programmering i Standard ML.}
%		\ressubitem{Retning af obligatoriske opgaver og eksamensopgaver.}
%		\ressubitem{Udarbejdning af vejledende løsninger.}
%	\end{ressubsec}
%
%    \begin{ressubsec}{Instruktor i kurset ``Algoritmer og Datastrukturer''}{Københavns Universitet}{April 2011 -- Juni 2011}
%		\ressubitem{Undervisning i grundlæggende algoritmer og datastrukturer.}
%		\ressubitem{Retning af obligatoriske opgaver.}
%		\ressubitem{Afholdelse af prøveeksaminer.}
%	\end{ressubsec}
%
%	\begin{ressubsec}{Instruktor i kurset ``Objektorienteret Programmering og Design''}{Københavns Universitet}{November 2009 -- Januar 2010}
%		\ressubitem{Undervisning i objektorienteret programmering og design i Java.}
%		\ressubitem{Retning af obligatoriske opgaver og eksamensopgaver.}
%		\ressubitem{Udarbejdning af obligatoriske opgaver og vejledende løsninger.}
%	\end{ressubsec}	
%
%	\begin{ressubsec}{Instruktor i kurset ``Introduktion til Programmering''}{Københavns Universitet}{September 2009 -- November 2009}
%		\ressubitem{Undervisning i funktionel programmering i Standard ML.}
%		\ressubitem{Retning af obligatoriske opgaver og eksamensopgaver.}
%		\ressubitem{Udarbejdning af vejledende løsninger.}
%	\end{ressubsec}

\end{ressection}

%%%%%%%%%%%%%%%%%%%%%%%%

\begin{ressection}{Tekniske kompetencer}
	\begin{reslist}{Programmeringssprog}
		\resitem{C\#, Python, Java, Standard ML, Perl, PHP, Mathematica, C, MIPS Assembler}
	\end{reslist}
	
	\begin{reslist}{Diverse sprog/formater}
		\resitem{CSS, HTML/XML, \LaTeX, SQL, YAML}
	\end{reslist}

    \begin{reslist}{Værktøjer}
		\resitem{Eclipse, VIM, Visual Studio}
		\resitem{Lidt erfaring med source control; mest git og darcs, lidt SVN.}
	\end{reslist}
\end{ressection}

%%%%%%%%%%%%%%%%%%%%%%%%

\begin{ressection}{Yderligere aktiviteter}

	\resitem{Vinder af Georg Mohr konkurrencen i matematik 2005 \& 2006.}
	
	\resitem{Deltager på det danske hold i Baltic Way konkurrencen i matematik 2005.}
	
	\resitem{Deltager for Danmark i the International Mathematical Olympiad 2006.}

    \resitem{Vinder af DM i Programmering 2010 og 2011.}

    \resitem{9. plads (2010) hhv. 7. plads (2011) i North Western European Regional Contest,
             kvalifikationsrunden til ACMs International Collegiate Programming Contest.}

\end{ressection}


\end{document}
