% Sebastian Paaske T�rholm's Resume
% Created: 7 Nov 2009
% Last Modified: 7 Nov 2009

\documentclass[11pt,oneside]{article}
\usepackage{geometry}
\usepackage[T1]{fontenc}

\pagestyle{empty}
\geometry{letterpaper,tmargin=1in,bmargin=1in,lmargin=1in,rmargin=1in,headheight=0in,headsep=0in,footskip=.3in}

\setlength{\parindent}{0in}
\setlength{\parskip}{0in}
\setlength{\itemsep}{0in}
\setlength{\topsep}{0in}
\setlength{\tabcolsep}{0in}

% Name and contact information
\newcommand{\name}{Sebastian Paaske T�rholm}
\newcommand{\addr}{Vinhaven 51, 1.mf, 2500 Valby}
\newcommand{\phone}{+45 22 85 56 23}
\newcommand{\email}{eckankar@gmail.com}


%%%%%%%%%%%%%%%%%%%%%%%%%%%%%%%%%%%%%%%%%%%%%%%%%%%%%%%%%
% New commands and environments

% This defines how the name looks
\newcommand{\bigname}[1]{
	\begin{center}\fontfamily{phv}\selectfont\Huge\scshape#1\end{center}
}

% A ressection is a main section (<H1>Section</H1>)
\newenvironment{ressection}[1]{
	\vspace{4pt}
	{\fontfamily{phv}\selectfont\Large#1}
	\begin{itemize}
	\vspace{3pt}
}{
	\end{itemize}
}

% A resitem is a simple list element in a ressection (first level)
\newcommand{\resitem}[1]{
	\vspace{-4pt}
	\item \begin{flushleft} #1 \end{flushleft}
}

% A ressubitem is a simple list element in anything but a ressection (second level)
\newcommand{\ressubitem}[1]{
	\vspace{-1pt}
	\item \begin{flushleft} #1 \end{flushleft}
}

% A resbigitem is a complex list element for stuff like jobs and education:
%  Arg 1: Name of company or university
%  Arg 2: Location
%  Arg 3: Title and/or date range
\newcommand{\resbigitem}[3]{
	\vspace{-5pt}
	\item
	\textbf{#1}---#2 \\
	\textit{#3}
}

% This is a list that comes with a resbigitem
\newenvironment{ressubsec}[3]{
	\resbigitem{#1}{#2}{#3}
	\vspace{-2pt}
	\begin{itemize}
}{
	\end{itemize}
}

% This is a simple sublist
\newenvironment{reslist}[1]{
	\resitem{\textbf{#1}}
	\vspace{-5pt}
	\begin{itemize}
}{
	\end{itemize}
}



%%%%%%%%%%%%%%%%%%%%%%%%%%%%%%%%%%%%%%%%%%%%%%%%%%%%%%%%%
% Now for the actual document:

\begin{document}

\fontfamily{ppl} \selectfont

% Name with horizontal rule
\bigname{\name}

\vspace{-8pt} \rule{\textwidth}{1pt}

\vspace{-1pt} {\small\itshape \addr \hfill \phone; \email}

\vspace{8 pt}




%%%%%%%%%%%%%%%%%%%%%%%%
\begin{ressection}{Uddannelse}

	\begin{ressubsec}{K�benhavns Universitet}{Bachelor}{Matematik med datalogi som bifag}
		\ressubitem{Igangv�rende, startet 2006.}
		\ressubitem{Gennemsnit indtil videre: 10,6.}
	\end{ressubsec}

\end{ressection}


%%%%%%%%%%%%%%%%%%%%%%%%
\begin{ressection}{Arbejde}

	\begin{ressubsec}{Instruktor i kurset "Introduktion til Programmering"}{K�benhavns Universitet}{September 2009 -- November 2009}
		\ressubitem{Undervisning i funktionel programmering i Standard ML.}
		\ressubitem{Retning af obligatoriske opgaver og eksamensopgaver.}
		\ressubitem{Udarbejdning af vejledende l�sninger.}
	\end{ressubsec}

	\begin{ressubsec}{Instruktor i kurset "Objektorienteret Programmering og Design"}{K�benhavns Universitet}{November 2009 -- Januar 2010}
		\ressubitem{Undervisning i objektorienteret programmering og design i Java.}
		\ressubitem{Retning af obligatoriske opgaver og eksamensopgaver.}
		\ressubitem{Udarbejdning af obligatoriske opgaver og vejledende l�sninger.}
	\end{ressubsec}

\end{ressection}

%%%%%%%%%%%%%%%%%%%%%%%%

\begin{ressection}{Tekniske kompetencer}
	\begin{reslist}{Programmeringssprog}
		\resitem{C\#, Java, Perl, PHP, Mathematica, Standard ML}
	\end{reslist}
	
	\begin{reslist}{Diverse sprog/formater}
		\resitem{CSS, HTML/XML, \LaTeX, SQL, YAML}
	\end{reslist}
	
	\begin{reslist}{V�rkt�jer}
		\resitem{Eclipse, VIM, Visual Studio}
		\resitem{Lidt erfaring med source control, mest SVN, git og darcs.}
	\end{reslist}
\end{ressection}

%%%%%%%%%%%%%%%%%%%%%%%%

\begin{ressection}{Yderligere aktiviteter}

	\resitem{Vinder af Georg Mohr konkurrencen i matematik 2005 \& 2006.}
	
	\resitem{Deltager p� det danske hold i Baltic Way konkurrencen i matematik 2005.}
	
	\resitem{Deltager for Danmark i the International Mathematical Olympiad 2006.}

\end{ressection}


\end{document}
